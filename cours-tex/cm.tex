\documentclass[12pt, a4paper]{article}
\usepackage[francais]{babel}
\usepackage{caption}
\usepackage{graphicx}
\usepackage[T1]{fontenc}
\usepackage{listings}
\usepackage{geometry}
\usepackage[colorlinks=true,linkcolor=black,anchorcolor=black,citecolor=black,filecolor=black,menucolor=black,runcolor=black,urlcolor=black]{hyperref}

% \usepackage{mathpazo} --> Police à utiliser lors de rapports plus sérieux

\usepackage{fancyhdr}
\pagestyle{fancy}
\lhead{}
\rhead{}
\chead{}
\rfoot{\thepage}
\lfoot{Martin Baumgaertner}
\cfoot{}

\renewcommand{\headrulewidth}{0.4pt}
\renewcommand{\footrulewidth}{0.4pt}

\begin{document}
\begin{titlepage}
	\newcommand{\HRule}{\rule{\linewidth}{0.5mm}} 
	\center 
	\textsc{\LARGE iut de colmar}\\[6.5cm] 
	\textsc{\Large R4ROM10}\\[0.5cm] 
	\textsc{\large Année 2022-23}\\[0.5cm]
	\HRule\\[0.75cm]
	{\huge\bfseries Opérateurs de télécom}\\[0.4cm]
	\HRule\\[1.5cm]
	\textsc{\large martin baumgaertner}\\[6.5cm] 

	\vfill\vfill\vfill
	{\large\today} 
	\vfill
\end{titlepage}
\newpage
\tableofcontents
\newpage
\section{CM 1 - 6 mars 2023}
\subsection{Traffic Engineering : MPLS-TE}
Premièrement, le MPLS permet de faire de la communication de label au lieu
de faire du routage. C'est bien plus rapide. C'est donc la principale raison 
pour laquelle on utilise le MPLS, comme par exemple avec le MPLS-VPN, 
pour interconnecter deux sites distants. 
Le protocol IP est un protocol de niveau 3, en fonctionnant en mode non 
connecté. 

Par l'ingénierie du trafic, on peut faire du routage de label, tout en garantissant
une qualité de service. \\

\textbf{MPLS-TE permet :}\\
\begin{itemize}
    \item la gestion des congestions imprévues
    \item une meilleure utilisation de la bande passante
    \item une meilleure capacité de planification\\
\end{itemize}

MPLS crée aussi des chemins LSP routés de façon explicite, ces LSP sont appelés
tunnels MPLS-TE. 

\subsection{Les composantes de la solution MPLS-TE}
Voici les 4 principales étapes : \\

\begin{itemize}
    \item 1. Découverte de la topologie du réseau : La première étape consiste à découvrir la topologie du réseau en utilisant des protocoles de routage de base tels que OSPF, ISIS ou BGP. Cette étape permet de cartographier le réseau et de déterminer les liens physiques entre les nœuds.
    \item 2. Sélection du chemin optimal : Après avoir découvert la topologie du réseau, la solution MPLS-TE sélectionne le chemin optimal pour acheminer les paquets de données. Cette sélection est effectuée en utilisant des algorithmes de calcul de chemin tels que CSPF (Constrained Shortest Path First), qui prennent en compte les contraintes de bande passante, de latence et d'autres paramètres.
    \item 3. Réservation de bande passante : Une fois le chemin optimal sélectionné, MPLS-TE réserve une quantité de bande passante pour ce chemin. Cette réservation de bande passante garantit que le chemin sélectionné ne sera pas congestionné et permet d'optimiser l'utilisation de la bande passante disponible.
    \item 4. Établissement de l'étiquette MPLS : Enfin, MPLS-TE établit une étiquette MPLS pour les paquets de données à transmettre. Cette étiquette indique le chemin optimal sélectionné pour acheminer les paquets et permet de contourner les traitements de routage de base pour optimiser les performances du réseau.
\end{itemize}

\newpage
\section{TD 1 - 8 mars 2023}
	\subsection{Exercice de compréhension}
		\subsubsection{Question 1}
		Le MPLS-TE est une technique de routage utilisée pour optimiser 
		le flux de données dans un réseau de télécommunications en fournissant 
		une qualité de service garantie et une bande passante réservée pour 
		les applications qui ont des exigences de performance élevées. 
		Elle utilise des étiquettes pour identifier les paquets de données et 
		déterminer le chemin optimal pour acheminer les données entre les 
		points d'entrée et de sortie du réseau.

		\subsubsection{Question 2}
		Le MPLS classique est utilisé pour acheminer les paquets de données 
		sur le chemin le plus court possible, tandis que le MPLS-TE est 
		utilisé pour optimiser le chemin des paquets en tenant compte des 
		exigences de performance et de bande passante des applications. 
		Le MPLS-TE permet de contrôler et de réserver la bande passante pour 
		des applications spécifiques, tandis que le MPLS classique ne prend
		pas en compte ces exigences de performance.\\

		TE : on utilise des tunnels 

		\subsubsection{Question 3}
		Un tunnel TE est un chemin de communication virtuel qui est établi 
		entre deux nœuds MPLS pour acheminer les paquets de données en 
		utilisant le MPLS-TE. Ce tunnel est créé en réservant une bande 
		passante et en établissant un chemin dédié pour les données qui 
		empruntent ce tunnel.\\

		Les protocols utilisés pour créer un tunnel TE sont :\\
		\begin{itemize}
			\item \textbf{RSVP-TE} (Resource Reservation Protocol - Traffic Engineering) : utilisé pour la réservation de bande passante et la sélection du chemin optimal pour acheminer les paquets de données.
			\item \textbf{LDP} (Label Distribution Protocol) : utilisé pour établir les étiquettes MPLS pour les paquets de données.
		\end{itemize}

		\newpage
		\subsubsection{Question 4}
		Le protocol IP ne suffit pas car il ne peut pas garantir une qualité
		de service et surtout, il ne peut pas répondre aux éxigences des 
		services de voix, vidéo d'autres applicatifs. 

		\subsubsection{Question 5}
		Voici les 4 étapes princaples pour l'utilisation du protocol MPLS-TE 
		: \\
		\begin{enumerate}
			\item \textbf{Distribution des informatioins} : quelles informations sont à récupérer ? 
			\item \textbf{Calcul du chemin} : Dynamique ou manuel ? 
			\item \textbf{Mise en place du chemin} : RSVP-TE ou CR-LDP ? 
			\item \textbf{Transférer le flux via le tunnel} : Autoroute, static ou policy ? 
		\end{enumerate}

		\subsubsection{Question 6}
		Pour établir les tunnels TE, les routeurs échangent des 
		informations de signalisation à l'aide de deux protocoles
		principaux : RSVP-TE (Resource Reservation Protocol - Traffic 
		Engineering) et LDP (Label Distribution Protocol).

		\subsubsection{Question 7}
		Ce sont les mêmes qu'à la questioin 6. 

		\subsubsection{Question 8}
		La détermination des chemins qui vont être utilisés par un tunnel 
		MPLS-TE est basée sur les exigences de qualité de service des 
		applications et sur la topologie du réseau. Le chemin optimal est 
		déterminé en fonction de la bande passante disponible, des contraintes 
		de qualité de service, de la topologie du réseau et des préférences 
		de routage.

		\newpage
		\subsubsection{Question 9}
		Il existe plusieurs options pour transférer les données sur un 
		tunnel MPLS-TE, selon les besoins de l'application et la configuration
		du réseau :\\
		\begin{itemize}
			\item Le mode Pipe (ou mode Uniform) : Dans ce mode, les paquets de données sont transférés sur le tunnel en utilisant la bande passante réservée pour le tunnel MPLS-TE. Tous les paquets qui empruntent le tunnel doivent respecter les paramètres de qualité de service (QoS) spécifiés lors de la création du tunnel.
			\item Le mode Short-Pipe : Dans ce mode, les paquets sont transférés sur le tunnel en utilisant l'étiquette MPLS associée au tunnel, mais la bande passante n'est pas réservée pour le tunnel. Les paramètres de QoS sont appliqués uniquement jusqu'à la sortie du dernier routeur d'entrée du tunnel, puis les paquets peuvent être acheminés de manière standard.
			\item Le mode Full-Pipe : Dans ce mode, les paquets sont transférés sur le tunnel en utilisant la bande passante réservée pour le tunnel MPLS-TE, mais les paramètres de QoS sont appliqués de manière dynamique tout au long du chemin du tunnel.
		\end{itemize}

		\subsection{Les attributs}
		\subsubsection{Question 1}
		


\end{document}